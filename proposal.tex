\documentclass[letter,11pt]{article}

\usepackage[english]{babel}
\usepackage[T1]{fontenc}
\usepackage[utf8]{inputenc}
\usepackage{lmodern}	% font definition
\usepackage{datetime}
\newdateformat{ymd}{\THEYEAR/\twodigit{\THEMONTH}/\twodigit{\THEDAY}}

\usepackage{color}
\usepackage{verbatim}
\usepackage{listings}
\lstset{ %
language=C++,                % choose the language of the code
basicstyle=\footnotesize,       % the size of the fonts that are used for the code
numbers=left,                   % where to put the line-numbers
numberstyle=\footnotesize,      % the size of the fonts that are used for the line-numbers
stepnumber=1,                   % the step between two line-numbers. If it is 1 each line will be numbered
numbersep=5pt,                  % how far the line-numbers are from the code
backgroundcolor=\color{white},  % choose the background color. You must add \usepackage{color}
showspaces=false,               % show spaces adding particular underscores
showstringspaces=false,         % underline spaces within strings
showtabs=false,                 % show tabs within strings adding particular underscores
frame=single,   		% adds a frame around the code
tabsize=2,  		% sets default tabsize to 2 spaces
captionpos=b,   		% sets the caption-position to bottom
breaklines=true,    	% sets automatic line breaking
breakatwhitespace=false,    % sets if automatic breaks should only happen at whitespace
%escapeinside={\%}{)}          % if you want to add a comment within your code
}

\usepackage{soul}
\usepackage{bm}
\usepackage{sectsty}
\sectionfont{\centering\LARGE}
\renewcommand{\thesection}{\relax}
\setcounter{secnumdepth}{-1}
\usepackage[compact]{titlesec}
\usepackage{mdwlist}
\usepackage{parskip}
\usepackage{graphicx}
\usepackage{framed}
\usepackage{array}
\usepackage{booktabs}
\usepackage{tabularx}
\usepackage{ltxtable}
\usepackage{multicol}
\usepackage[top=1.8cm,bottom=2.0cm,right=1.9cm,left=1.9cm]{geometry}
\usepackage{pdflscape}
\usepackage[numbers]{natbib}

\usepackage{amsmath}
\usepackage{amsthm}
\usepackage{latexsym}
\usepackage{cancel}

\usepackage[urw-garamond]{mathdesign}

\usepackage{tikz}
\usepackage{pgfplots}

\usetikzlibrary{arrows,shapes,calc}

\usepackage{url}
\usepackage[ps2pdf,breaklinks=true,bookmarks=true,bookmarksopen,bookmarksopenlevel=2,pdfpagelayout=OneColumn,pagebackref=true]{hyperref}
\usepackage{breakurl}
\usepackage{makeidx}

%\yyyymmdddate

\hypersetup{
    bookmarks=true,         % show bookmarks bar?
    unicode=false,          % non-Latin characters in Acrobat’s bookmarks
    pdftoolbar=false,        % show Acrobat’s toolbar?
    pdfmenubar=true,        % show Acrobat’s menu?
    pdffitwindow=true,     % window fit to page when opened
    pdfstartview={FitBV},    % fits the width of the page to the window
    pdfauthor={David Dalrymple},     % author
    pdfcreator={David Dalrymple},   % creator of the document
    pdfproducer={David Dalrymple}, % producer of the document
    pdfnewwindow=true,      % links in new window
    colorlinks=true,       % false: boxed links; true: colored links
    linkcolor=blue,          % color of internal links
    citecolor=magenta,        % color of links to bibliography
    filecolor=magenta,      % color of file links
    urlcolor=cyan           % color of external links
}


\renewcommand*{\backref}[1]{}
%\renewcommand*{\backrefalt}[4]{%
  %\ifcase #1 %
%(Not cited.)%
%\or
%(Cited on page #2.)%
%\else
%(Cited on pages #2.)%
%\fi
%}

\newcommand{\attrib}[1]{\nopagebreak{\raggedleft\footnotesize #1\par}}
\newcommand{\todo}[1]{\textcolor{lightgray}{\textit{<<#1>>}}}
\newcommand{\tbc}{\begin{center} \todo{to be completed} \end{center}}

\setlength{\parskip}{0.3cm plus3mm minus1mm}
\setlength{\parindent}{0cm}
\setlength{\columnsep}{7mm}
\setlength{\premulticols}{1cm}
\setlength{\postmulticols}{1cm}


\begin{document}

\begin{center}
\parbox[b]{3cm}{\hfill Applied Math 147 \hfill} \hspace{1cm} Project Proposal: \textbf{The Dynamics of Language} \hspace{1cm} \parbox[b]{5cm}{\hfill \textit{David Dalrymple and Yiren Lu} \hfill} \\[-3mm]
\rule{\textwidth}{0.4pt}
\end{center}



\section{Goals}

Language is one of the most complicated phenomena in everyday life, and by far
the most common means by which human beings interact with each other. Viewed as
a dynamical system, it is fascinating that spatial (written) and temporal
(spoken) patterns can couple brain states so effectively! A key capability for
this process is the remarkable properties of the mammalian cochlea, the organ
that transduces the mechanical vibrations of sound into neural impulses. Our
goals in this project are \textbf{to understand quantitatively the function of
the cochlea} in the context of language, and \textbf{to study the
characteristics of the local parameter space}, with an eye toward the evolution
of the cochlea's capabilities (on the ``slow time-scale'' of evolution\cite{manley72}).

\section{Plan}

We will begin by developing a computer simulation of the basilar membrane of
the cochlea, the organ responsible for mechanical spectral analysis of incoming
auditory stimulation\cite{nilsen99,ruggero97}. The basilar membrane can be
approximated as a series of coupled mass-spring oscillators with different
resonant properties \cite{hubbard06,hubbard96}, as a cascade of filters
\cite{linggard89}, using finite element methods \cite{skrodzka05}, or a variety
of other techniques. Our first task will be to evaluate each of these
approaches for ease of implementation, biological accuracy, and mathematical
elegance (or ease of analysis). After choosing and implementing a model, we
will test the simulation by coupling the boundary conditions with recordings of
actual speech. Time permitting, we will pursue some of the following
extensions:

\begin{itemize}
	\item analyze the sensitivity of cochlear functions to various tuning parameters \cite{yates90} and identify bifurcations
	\item develop a neural model of phoneme classification \cite{mesgarani08}, which will be coupled to the output of the
basilar membrane model
	\item attempt a mechanical cochlea \cite{hubbard06,keolian97}
\end{itemize}

\bibliographystyle{abbrvnat}
\bibliography{am147}


\end{document}
